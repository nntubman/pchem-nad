% delete the .aux file if "!Package babel Error: You haven't defined the language ENGLISH yet."
\documentclass[aip,jcp,numerical,reprint]{revtex4-1}
\usepackage{graphicx}  % needed for figures
\usepackage{threeparttable} % for annotated table
\usepackage{siunitx} % for decimal alignment in tables
\usepackage{bm,amssymb,amsmath} % for math
\usepackage[bookmarks]{hyperref} % for reference links
\usepackage[svgnames]{xcolor} % to define color names
\definecolor{Blue}{RGB}{0 0 128}
\hypersetup{ %options-for-appearance-of-links-in-hyperref
	colorlinks= true,
    allcolors = Blue
}
\usepackage{tikz}
\usetikzlibrary{calc} % for tikz calculations
\usetikzlibrary{arrows,decorations.markings} % make arrow head bigger
\usepackage{color,soul} % for reviews
\begin{document}

\title{Improving wave functions for non-adiabatic calculations with quantum Monte Carlo}
\author{NMT, YY, SHS, DMC}
\affiliation{Department of Physics, University of Illinois, Urbana, Illinois 61801 USA}

\begin{abstract}
Sstate of the art calculations for non-adiabatic simulations is a relatively unexplored field, especially in comparison to the number of methods and simulations that strictly make the Born-Oppenheimer approximation.  Various algorithms can be applied that capture the phenomenology of non-adiabatic effects beyond the Born-Oppenheimer approximation, however except in very few cases its hard to determine how accurate such calculations are.  One approach that has been discussed in the literature in varying context in the last few years is a focus on improving ground state wave functions.  This has come in several different forms, which include studies that focus on the form of the wave function, highly approximate methods that focus on correlations between electrons and ions, and more exact methods that work for model Hamiltonians or small systems, which includes density matrix renomarlization group, quantum monte Carlo, explictly correlated gaussians, as well as some analytically exact results.  Here we briefly review various approachs, with a focus on our previous quantum Monte Carlo result.  In particular we focus on the CH molecule for which are previous results showed a big non-adiabatic effect.  We find that by using a new wave function ansatz that half of this non-adiabatic energy goes away, but it still remains one of the most non-adiabatic molecules we have tested.


\end{abstract}
\maketitle

\section{Introduction}
The Born Oppenheimer is considered one of the most accurate approximations ever made.  However, the break down of the Born Oppenheimer approximation can lead to interesting new physics in many cases, and its impact may still not be fully appreciated due to the lack of theoretical studies and methods that can go beyond the Born Oppenheimer approximation accurately.    The absorption spectrum from the ISM is one such case where highly accurate energies without the Born-Oppeneheimer approximation is needed of both ground and excited state quantum systems.   The ISM features many absorption peaks  from molecules that have yet to be identified.  Highly accurate recent experiments have been able to show that peaks corresponding to ionized C60 are present in the interstellar medium.  However, most molecule remain unidentified, and theoretical calculations have been lacking due to lack of methods that can actually address these problems to the level of accuracy required.  

To make progress in this direction, we developed a formalism and calculated ground state simulations of several small molecules in which the non-adiabatic effects were not approximated in any way beyond what is the fixed node approximation for a wave function that includes both electrons and ions.  We demonstrated that for most systems, which were limited to atoms and diatomic molecules, that for the most part that non-adiabatic effects were smaller than 10$^{-4}$ mHa, with two exceptions.    However, our results are only as good as the nodes for the wave function used for those simulations, and it remains an open question as to how large the non-adiabatic effects in those systems are. 
In this work we review progress we have made in improving the accuracy of  ground state of calculations beyond the Born-Oppenheimer approximation, and we demonstrate a new wave function ansatz that is more accurate than in previous work.


\section{Fixed-Node Diffusion Monte Carlo (FN-DMC)}
Diffusion Monte Carlo~\cite{Anderson_DMC,lester1,Stuart_Review,Needs_Review,Needs_Old_Review,QMC_Review} is a projector method that evolves a trial wave function in imaginary time and projects out the ground-state wave function. For practical simulations of fermions, the fixed-node approximation is introduced, which depends only on the set of electronic positions where a trial wave function is equal to zero.  From a theoretical perspective, a wave function that jointly describes electrons and ions simultaneously is implicitly included in the Born-Oppenheimer approximation, and are explicitly taken into account when harmonic or anharmonic effects are calculated.  Writing down wave functions that go beyond the Born-Oppenheimer approximation are not hard to generate, but finding a good form for them is something that has really generated  interest in the last few years.  In consideration of diffusion Monte Carlo, treating non-adiabatic wave functions requires minimal changes, with the main differences are seen in evaluating a different form of the trial wave function, and  evaluating the kinetic energy term for which there is a mass dependence on the particles.  This actual approximation in such an FN-DMC is a joint nodal surface that depends on the coordinates of both the electrons and ions simultaenously, which is different from other standard methods that have been used to treat Hamiltonians beyond the Born-Oppenheimer approximation, and we have recently shown that we can generate high-quality nodal surfaces for a range of systems that include full electron-ion wave functions. 

If the trial wave function has the same nodal surface as the exact ground-state wave function, FN-DMC will obtain the exact ground-state energy.  Approximate nodal surfaces can be generated through optimization of the full wave function. Such approximate nodal surfaces have been tested and validated on a wide range of systems, and consistently provide an excellent approximation of the exact ground-state energy, comparable to the state of the art in \textit{ab initio} simulations.~\cite{grossman1} In addition, the energies generated with FN-DMC are variational with respect to the ground-state energy.


Outside of recent calculations done by us, there has been little work in treating non-adiabatic simulations with QMC and limited development in producing wave functions for realistic systems beyond a few variational parameters.  For instance Chen and Anderson \cite{anderson1} performed the most accurate QMC calculation on  H$_{2}$ with a wave function  specified completely in terms of relative coordinates and only a few variational parameters. which are different from typical static ion problems that use basis sets that depend on the position of the ions.   The wave functions used in these non-adiabatic simulations are optimized by simultaneously treating all correlations between all the particles simultaneously.
The  H$_{2}$ molecule is a well studied electron-ion system in DMC, as there is no sign problem and the system can be calculated without any biases. The wave function for the Chen-Anderson DMC result  is a product of four terms, two of which are electronic orbital that consists of a sum of two exponentials.  The  other two terms are Jastrow terms for the electron-ion and electron-electron correlations, which captures all the wave function cusps.  All terms used in the wave function only depend on relative distances and are rotationally symmetric, thus the ions and electrons are free to rotate and translate in space.  

For a system such as H$_{2}$, the accuracy is defined as an absolute accuracy, as there are no systematic biases, and thus the best DMC results can be considered those that have the smallest error bar.  In terms of efficient simulation, one must considered the variance of the local energy of a trial wave function and the computational expense needed to sample a trial wave function.  The Chen-Anderson energy for the H$_{2}$ molecule has an the error bar slightly smaller than 10$^{-5}$ (a.u.).  While this a highly accurate result, there are several other non-QMC results, including one by Bubin-Adomwicz, that are thought to be accurate to many more digits, even though there is some inherent bias due to a finite basis set.  The Bubin-Adomwicz result uses a basis of explicitly correlated gaussian (ECG) which produces many terms that require an exponential amount of computational cost with increasing system size. 
In larger systems, it is in many cases too expensive to converge the wave functions, with either method, to such high accuracy.  In FN-DMC there is an error associated with treating fermions, and with the ECG technique, the computational cost grows too quickly to converge the wave function, For large systems, full CI would also become prohibitively expensive.  

For static ion simulations, FN-DMC  it is a highly accurate technique, even for large systems despite the fixed-node approximation.  The question becomes how to create a trial wave function to use in FN-DMC for electron-ion methods.
Using recent improvements in generating highly accurate wave functions that can be optimized with FN-DMC, we have shown in previous work that highly accurate wave functions can be generated for electron-ion systems.  However, for applications of spectroscopy, it remains an open question if we can push our simulations to be accurate enough to solve questions related to identifying ISM peaks for instance.


\subsection{Electron-Ion Wave Function}
 There are several forms in which one might try to build a wave function for electron-ion systems.  Other than the type of wave function used in Chen-Anderson, the other well known wave function for electron-ion systems is a single determinant of plane waves for both the electrons and ions, which doesn't change as the ions move.  The cusps are enforced by a Jastrow, which does move with the electron and ions, but the accuracy of the calculations are limited by the nodes, which as plane waves, do not spend on the physical interactions between the electrons and ions.  
 

We expect the best wave functions that one can use for these calculations is a full position dependent electronic wave function, that is optimized at each ionic position.  Ideally one would like to optimize all wave function parameters at each ion position.  For many ion problems, this would require each walker to have its own wave function and because of the stochastic nature of optimizing wave functions, the wave function is not guaranteed to be consistent as the ions move around.  Static Jastrow parameters have been used in CEIMC type calculations, which would alleviate that problem, however a wave function call would have to be made to an external program for each walker for each  ion configuration, which is more computationally expensive and technically difficult than should be considered for our first electron-ion wave functions.   
Several strategies can be implemented to use such wave functions in FN-DMC calculations, which are applicable to a wide range of problems having a combination of fixed and quantum nuclei.  
%Before discussing the form considered in this work, we note that there are many possibilities for creating such wave functions for use in QMC, and there is certainly room for a l significant development in this direction.  
We consider three different wave function forms that are progressively more accurate as follows:
\begin{align}
\Psi(r,R) =& e^{J(r,R)}\phi(R)\sum_{i}\alpha^{*}_{i} D_{i}(r) \label{eqn:wfs1}\\
\Psi(r,R) =&e^{J(r,R)}\phi(R)\sum_{i}\alpha^{*}_{i} D_{i}(r,R^{*}) \label{eqn:wfs2}\\
\Psi(r,R) =& e^{J(r,R)}\phi(R)\sum_{i}\alpha^{}_{i} D_{i}(r,R), \label{eqn:wfs3}
\end{align}
where $r$ refers to the coordinates of all the electrons and $R$ to those of all the ions.  $J(r,R)$ is the Jastrow term which involves variational parameters that correlate the quantum particles and additionally  enforce cusp conditions in the wave function.  $\phi(R)$ is the nuclear part of the wave function. The final terms correspond to determinants $D$ and the corresponding coefficients $\alpha$.    The $*$ denotes how these terms are evaluated, as will be discussed. 

The nuclear part of the wave function is chosen to be a simple product of gaussian functions over each nucleus pair: 
\begin{align}
%\phi(R) \propto \prod_{\substack{i \\ i<j}} \exp\left[-a_{ij}\left(|R_{i}-R_{j}|-b_{ij}\right)^2\right], 
\phi(R) \propto \prod_{\substack{i \\ i<j}} e^{-a_{ij}\left(|R_{i}-R_{j}|-b_{ij}\right)^2}, 
\end{align}
where $a$ and $b$ are optimizeable parameters. In our calculations $a_{ij}$ has only a single optimized value $a$, and for $b_{ij}$ we use the Born-Oppenheimer equilibrium distance between the species involved.

The terms in these wave functions involve very specific calculations that are performed and optimized in both quantum chemistry codes and quantum Monte Carlo codes.  
%The differences between the wave functions is in how we  compute the determinant of the electrons $D_{i}$.  
The determinant terms, $\alpha_{i}^{*}D_{i}(r) $, $\alpha_{i}^{*}D_{i}(r,R^{*}) $, and $\alpha_{i}^{}D_{i}(r,R)$ differ based on how we optimize the determinant coefficients $\alpha$ and how we parameterize the evaluation of the determinants based on the ion coordinates $R$.   

The  wave function in Eq.~\eqref{eqn:wfs1} is the least accurate of the three wave functions and has a fixed determinant regardless of where the ions are.  The term $\alpha^{*}$ indicates that the determinant coefficients have been optimized at the equilibrium geometry.  
%Practically this is implemented by having the determinant part of the wave function independent of the ion positions.  
Both the ionic part of the wave function ($\phi$) and the Jastrow depend on the ion positions, which is important as the Jastrow maintains the cusps between all the quantum particles.  

%This form of the wave function has previously been used for large-scale simulations of metallic hydrogen~\cite{ceperley3,natoli1,natoli2}.  %As an ansatz the wave function in this work  used a single determinant of plane waves for both the electrons and ions, with only the electron-ion Jastrow to capture the interactions between the two species.  
The problem with this type of wave function is that the accuracy is limited by the electronic nodes, which do not depend on the ion positions.
This may be a good approximation for condensed matter systems, but in general  the determinant should depend on the ionic coordinates. 
%This is especially important for heavier atoms, beyond helium, in which the core orbitals should track the ion positions. 

The wave function in Eq.~\eqref{eqn:wfs2} fixes many of the problems of the previous wave function.  
%is improved over the previous wave function in that the nodes of the determinant part of the wave function depend on the ion positions.  
The $\alpha^{*}$ indicates that the determinant part of the wave function is optimized for the equilibrium ion positions, as in the previous wave function, but the term $R^{*}$ signifies that the determinant  depends on the position of the ions through the basis set.  Basis sets in molecular calculations are generally constructed from local orbitals centered around the atoms.  In these calculations a single particle orbital is written as $\theta(r) = \sum_{ji}\gamma_{j}(r-R_{i})$, where \textit{i} is an index for an ionic center, and \textit{j} is an index for a basis set element.  
In this form, wave functions depending on the ion positions are straightforward to create and optimize,
%Wave functions in this form provide a straightforward way  of creating and optimizing wave functions that depend on the ion positions.  
%For systems that are not strongly non-adiabatic, the wave function will not significantly change depending on the ion positions, and thus this wave function ansatz should be good, especially for describing the electronic nodes.
%  There are some problems with this form of the wave function, as for instance single particle orbitals can have directional dependence, such as in a covalent bond.  
but difficulties may arise with the possible directional dependence of the single body orbitals, such as in covalent bonds. This can be addressed with directionally dependent Jastrows, but we go further than this, as will be discussed.  This form of the wave function is similar to the wave function used in Ref.~\cite{ceperley3} for the molecular hydrogen phases.   They are not quite the same, however, as the electronic orbitals and ionic orbitals were centered around fixed positions and thus the electronic orbitals did not explicitly track the ion positions.   A few of the simulations did have the electrons track the centers of the hydrogen molecules as they changed position.

Eq.~\eqref{eqn:wfs3} represents what we expect to be the best wave function considered here, since it has explicit dependence on the ion positions for the single particle orbitals and the determinant coefficients. Essentially this amounts to recalculating a wave function from scratch each time the ion positions are changed.  This would significantly increase the computational cost of these simulations as well as cause many technical challenges. 
% Ideally one would like to optimize all wave function parameters at each ion position.  
%One way to implement this would be to assign each walker its own separate wave function. Because of the stochastic nature of optimizing wave functions, it is possible that a wave function may not be entirely single valued as the ions move around.  Static Jastrow parameters have been used in coupled electron ion Monte Carlo calculations, which would alleviate this problem; however a wave function call would still have to be made to an external program for each walker and for each ion configuration.  

%Equation \ref{eqn:wfs3}, is a very accurate form of the wave function in which all variational parameters, except those in $\phi$, are full optimized at all ion positions.  This is a very expensive form of the wave function to maintain, and its not entirely clear in its most flexible form how to make it completely consistent, since the wave functions parameters are stochastically optimized.  Additionally this form requires each walker to have its own wave function, which can be technically challenging during a FN-DMC run.  Thus we focus on wave function in equation \ref{eqn:wfs2} in this work.

\section{Improving wave functions}
Previously we have explored wave functions of the form~\eqref{eqn:wfs2}.  We studied several different atomic and molecular systems, and we found that in the case of BH and CH that the non-adiabatic effects were  significantly larger than 10$^{-4}$, which was not true for any of the other atoms and molecules.  To determine the non-adiabatic effects, we broke down the energy of our systems into different components, which includes the static atom energies, the zero point energy (ZPE), the diagonal Born-Oppenheimer energy.   Everything left over we consider to be the non-adiabatic energy.   The first three can be approximated very easily in various quantum chemistry packages, but estimating the non-adiabatic energy is something that is not well studied, and our quantum Monte Carlo results were among the first.  Our non-adiabatic energy was over 1mHa for the CH atom, which is quite large in comparison to all the other systems we consider, and was larger than what was expected from experiment.  However there was only one experimental result and we wanted to consider this calculation in more depth due to the nature of the fixed node approximation in estimating non-adiabatic effects.

Within the FN-DMC formalism, the fixed node approximation is generally going to overestimate the non-adiabatic effects.   Generally we believe we can optimize static ion calculations better than the electron-ion wave functions, thus we expect the error in the energy difference between these calculations to overestimate the energy do to purely non-adiabatic effects.  It should be noted that this expectation isn't always true, as seen in benchmarks comparisons of (Be,Be+,B,B+,C+) in comparison to highly converged ECG results, sometimes our non-adiabatic simulations are as good or more accurate than our static ion simulations.  Regardless, when large non-adiabatic effects are soon, it is important to consider how big the fixed-node approximation really is.  In fact, all the simulations we have previous performed used a particular type of nodal structure, which we termed the dragged node approximation.
 
To understand the dragged node approximation  it is important to consider the different nodal structures in the wave functions given by Eqs.~\eqref{eqn:wfs2} and \eqref{eqn:wfs3}.  In Eq.~\eqref{eqn:wfs3}, the nodes are defined by the determinant that is calculated at each position in space, but in Eq.~\eqref{eqn:wfs2}, we use the determinant defined at the equilibrium geometry, and then drag those nodes around through the basis set dependence.   The dragged node approximation is completely variational when used in VMC and FN-DMC.  
For systems that do not show strong non-adiabatic behavior, we expect this to be an excellent approximation as this works best when a wave function does not change sharply as the ions change positions.   Even though CH probably is not what one would call a non-adiabatic system, the energy due to non-adiabatic effects was large enough that we wanted to improve the wave function we used for the simulation.    Since the CH molecule is made up for only two atoms, it is possible to use a wave function of type \eqref{eqn:wfs3} as discussed below.   A wave function of this type can have significantly more complicated dependence of the nodes based on the ion positions than used in the dragged node approximation.

To understand how strong the non-adiabatic effects are in CH, we wanted to consider a wave function in which there were more variational degrees of freedom for the nodes.  As described previously, the wave function \eqref{eqn:wfs3} is much more general than what we included in our previous studies, but is harder to generate.  Specifically it is not feasible to do a full wave function evaluation for each new configuration of the ions.  However, since there are only two ions in this molecule, it is feasible to precompute and optimize wave functions at different distances before during a full FN-DMC calculatio, and use the precomputed calculations in order to interpolate the wave function for all the relevant ion distances.  There are several different ways this can be done.   The first wave function one might consider is to actually grid up the distance between the ions, and calculate a fully optimized electronic wave function at each grid point.  Then one can evaluate the electronic wave function at each grid point, and use a weighting scheme to determine the actual electronic wave function given a distance between the ions.  This would be multiplied by a purely ionic wave function, as shown in \eqref{eqn:wfs3}.  Due to issues of wave function smoothness, we found it easier to just consider a wave function with the electronic determinant coefficients interpolated instead.

To be specific, we consider the following procedure.  At the equilibrium position, optimize an electronic wave function, which includes all determinant coefficients and a Jastrow.   As a large distance,  but still in the region in which the ions sample, we optimize an electronic wave function, but only the determinant coefficients.  We fix the Jastrow to be the same as the one we optimized at the equilibrium position.  We then take all the determinant coefficients for the two different wave functions, and fix a second order polynomial to them (there are general a few hundred of these), for which we then use to interpolate the multi-determinant wave function for all ion positions.  In this way we go beyond the dragged node approximation, as the coefficients of the determinants are allowed to change rapidly with ion distance, and can capture complicated dependences of the nodes.

\section{Results and Discussion}

Our results are quite interesting, when we apply this wave functions to the CH molecule.  Our previous results showed a non-adiabatic energy of roughly 2mHa.   Our new results show a non-adiabatic energy of roughly 1mHa.  While this is still not a definitive result, this is what we would have expected for the case in which the non-adiabatic results are non-negligible for the system.  It is evident that when a system has somewhat sizeable non-adiabatic effects that more effort is going to be needed in generating accurate wave functions.  The improved wave functions will always lower the energy and likely reduce the non-adiabatic energy.  The non-adaibatic energy is still large however, which is consistent with the fact that we had to use a wave function of the type seen in equation \eqref{eqn:wfs3}.



\section{Conclusion}

In this work, we demonstrated that for diatomic systems, there are ways of generating wave functions beyond what has been done in previous quantum Monte Carlo work.  These wave functions are generated from highly accurate static quantum chemistry techniques, from which nodes are derived that are better than those of the dragged node approximation.  We have been explicitly interested in the CH molecule, due to our previous results that show some what large non-adiabatic energy, and in this work we improve upon our estimate of this energy.  Further calculations are possible to improve our results here, such as release node calculations, however, real progress for larger systems will require even more clever scheme to improve wave functions.   This is however is a further step to improve the accuracy of non-adiabatic simulations to start approaching the accuracy needed to compare to spectroscopic results and predictions for the peaks in the ISM.


\section{Acknowledgment}
 This work was supported by the U.S. Department of Energy (DOE) Grant No. DE-FG02-12ER46875 as part of the Scientific Discovery through Advanced Computing (SciDAC) program. NT and DC were supported by DOE DE-NA0001789. S.H.-S. acknowledges support by the National Science Foundation under CHE-13-61293. J.T.K. was supported through Predictive Theory and Modeling for Materials and Chemical Science program by the U. S. Department of Energy Office of Science, Basic Energy Sciences (BES). We used the Extreme Science and Engineering Discovery Environment (XSEDE), which is supported by the National Science Foundation Grant No. OCI-1053575 and resources of the Oak Ridge Leadership Computing Facility (OLCF) at the Oak Ridge National Laboratory, which is supported by the Office of Science of the U.S. Department of Energy under Contract No. DE-AC05-00OR22725.

\bibliography{ref}
\end{document}
